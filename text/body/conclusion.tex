\section{Conclusion} \label{sec:conclusion}

Upwards trends in technological capacity for massively parallel and distributed computation are continuing to profoundly redefine the scope of scientific questions and model systems that can be investigated using \textit{in silico} methods.
As these systems scale, there become challenges with data overload in running and observing them.
Sampling based approaches and approximations provide one possible approach to solving this issue.

In a parallel vein, advances in high throughput sequencing technologies and phylogenetic inference algorithms over nucleotide data are making it possible to work with larger and larger trees.
These have brought into realization phylogenies with millions or tens of millions of tips.
On the extreme end, there was recentlly proof-of-concept work done using procedurally generated virtual sequence data related to barcoding procedures that built a 333 million tip tree \citep{konno2022deep}.



Reflect on the fact that phylogeny data in biology is also getting larger scale due to better sequencing and reconstruction algorithms; there’s a larger question about what do you do with a multimillion/billion tip tree and how do you do it in an efficient manner
Infrastructure for storing, measuring, manipulating, and visualizing phylogenies is a key bottleneck.
Reference taxonium \citep{sanderson2022taxonium}
Reference Niema projects on scalable phylogeny representations \citep{moshiri2025compacttree,moshiri2020treeswift}.
https://niema.net/ has some recent papers (one on a C++ library and one on a python library, both of which we should cite; and we should cite the other scalable phylogeny software/references that Niema cites)

Alife data standard and dataframe-oriented representations for phylogeny
Which combine well with Numba, Numpy, Polars, and Pandas
Which is kind of how Ape represents phylogenies in R
There are prototype implementation in hstrat utils
Future Work:
Some work could use the algorithm and easy-to-use software for testing hypothesis on their own simulations
Open source software:
Ie hstrat is open source
So you can do it too!

