\begin{abstract}
Agent-based simulation platforms play a key role in enabling fast-to-run evolution experiments that can be precisely controlled and observed in detail.
Availability of high-resolution snapshots of lineage ancestries from digital experiments, in particular, is key to investigations of evolvability and open-ended evolution, as well as in providing a validation testbed for bioinformatics method development.
Ongoing advances in AI/ML hardware accelerator devices, such as the 850,000-processor Cerebras Wafer-Scale Engine (WSE), are poised to broaden the scope of evolutionary questions that can be investigated \textit{in silico}.
However, constraints in memory capacity and locality characteristic of these systems introduce difficulties in exhaustively tracking phylogenies at runtime.
To overcome these challenges, recent work on hereditary stratigraphy algorithms has developed space-efficient genetic markers to facilitate fully-decentralized estimation of relatedness among digital organisms.
However, in existing work, compute time to reconstruct phylogenies from these genetic markers has proven a limiting factor in achieving large-scale phyloanalyses.
Here, we detail an improved trie-building algorithm designed to produce reconstructions equivalent to existing approaches.
For modestly-sized 10,000-tip trees, the proposed approach achieves 300-fold speedup versus existing state-of-the-art.
Finally, using 1 billion genome datasets drawn from WSE simulations encompassing 954 trillion replication events, we report a pair of large-scale phylogeny reconstruction trials, achieving end-to-end reconstruction times of 2.6 and 2.9 hours.
In substantially improving reconstruction scaling and throughput, presented work establishes a key foundation to enable powerful high-throughput phyloanalysis techniques in large-scale digital evolution experiments.
\end{abstract}
