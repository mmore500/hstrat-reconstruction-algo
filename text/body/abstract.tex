\begin{abstract}

Agent-based simulation platforms play a key role in enabling fast-to-run evolution experiments that can be precisely controlled and observed in detail.
Availability of high-resolution snapshots of lineage ancestries from digital experiments, in particular, is key to investigations of evolvability and open-ended evolution, as well as in providing a validation testbed for bioinformatics method development.
Ongoing advances in hardware accelerator devices for high-performance computing, such as the 850,000-processor Cerebras Wafer-Scale Engine (WSE), are poised to broaden the scope of evolutionary questions that can be investigated \textit{in silico}, but memory and bandwidth constraints introduce difficulties in exhaustively tracking phylogenies.
Recent work on hereditary stratigraphy algorithms has developed space-efficient genetic markers to facilitate fully-decentralized estimation of relatedness among digital organisms.
However, computational intensity of reconstructing phylogenies from these genetic markers has proven a limiting factor in large-scale phyloanalyses.
Here, we detail an improved trie-building algorithm designed to produce reconstructions equivalent to existing approaches, and report benchmarking and validation experiments assessing its properties.
For modestly-sized 10,000-tip trees, the proposed approach achieves 300-fold speedup versus existing state-of-the-art.
In scaling analysis, workloads up to 10 million genomes maintain a promising 87\% of marginal throughput rate.
Finally, using 1 billion genome datasets drawn from WSE simulations encompassing 954 trillion replication events, we report a pair of large-scale phylogeny reconstruction trials, achieving end-to-end reconstruction times of 2.5 and 3.5 hours.
In substantially improving reconstruction scaling and throughput, presented work stands to help enable incorporation of powerful phyloanalysis techniques requiring large sample size in decentralized digital evolution experiments.
\end{abstract}
