\begin{abstract}

Digital evolution enables fast experiments that can be precisely controlled and observed in detail.
This makes them useful both as a testbed for developing and validating statistical methodologies for phylogeny-based analyses.
These platforms are also of interest in studying abstract concepts like evolvability and open-ended evolution, where the ability to explore parameter regimes outside thoseof natural life is relevant.
Phylogeny-based analyses leveraging bioinformatics infrastructure are an important tool for this use case.
Large-scale simulations are important to both of these use cases.
Ongoing advances in hardware accelerator devices such as the Cerebras Wafer-Scale Engine have great potential to enable very large-scale simulations, but memory and bandwidth constraints introduce difficulties in exhaustively tracking phylogenies.
Recent work has developed space-efficient mechanisms for heritable genetic markers to facilitate fully-decentralized approximate tracking of ancestry over evolving populations called hereditary stratigraphy.
However, the process of reconstructing phylogenies from genetic markers has been a bottleneck to efficiently carrying out large-scale analyses.
Here, we detail an improved phylogeny reconstruction algorithm and report empirical benchmarking experiments to assess its scalability.
The proposed approach achieves TODO-fold speedup for constructing trees of size TODO, and, additionally, exhibits a more benign scaling relationship trajectory than the naive approach.
In a set of large-scale trials using data sampled from a wafer-scale simulation encompassing TOOD quadrillion agent evaluations, we find build times for a 1 billion genome tree between 2.5 and 3.5 hours.
The presented work provides a lynchpin to enable work in large-scale evolution simulations in a memory-efficient manner that harnesses large numbers of compute cores.
This will allow experiments that require building very large phylogenetic trees, or high-throughput experiments that require building many smaller trees such as sampling-based methods or gene-tree analyses.
To facilitate these use cases, a reference implementation is provided as a Python package and a standalone CLI interface.

\end{abstract}
