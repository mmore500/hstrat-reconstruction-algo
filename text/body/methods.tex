\section{Methods} \label{sec:methods}

\subsection{The Algorithm} \label{sec:algorithm}

We present an iterative algorithm that processes each organism one at a time, inserting it into the tree at the appropriate location. The functions \textsc{ReconstructTree} and \textsc{TreeInsert} are, for the most part, conventional for trie-building algorithms \citep{crochemore2009trie}. However, the distinguishing factor of our algorithm is how it deals with missing information, through the \textsc{ConsolidateTree} process.

\begin{algorithm}[h]
    \caption{the main algorithm for creating a phylogenetic tree}
    \begin{algorithmic}[1]
        \Require a list of organisms $O$ in ascending order by generations elapsed
        \Function{ReconstructTree}{$O$}
            \State $T \gets$ an empty tree
            \For {$o \in O$} 
                \State $\textsc{TreeInsert}(T, o)$
            \EndFor
            \State \Return $T$
        \EndFunction
    \end{algorithmic}
\end{algorithm}

\begin{algorithm}[h]
    \caption{the algorithm for inserting organisms into the tree}
    \begin{algorithmic}[1]
        \Require a phylogenetic tree $T$ and organism $o$
        \Function{TreeInsert}{$T, o$}
            \State $n \gets$ root of $T$
            \For{each rank-differentia pair $(r, d) \in o$}
                \If{$r >$ the rank of $n$'s children}
                    \State $\textsc{ConsolidateTree}(T, \dotsc)$
                \EndIf
                \If{$\exists c \in \operatorname{children}(n) \text{ s.t.} \operatorname{differentia(c) = d}$}
                    \State $n \gets c$
                \Else 
                    \State create a new child $c'$ branching off $n$ where $\operatorname{differentia}(c') = d$
                    \State $n \gets c'$
                \EndIf
            \EndFor
        \EndFunction
    \end{algorithmic}
\end{algorithm}

\subsection{Software and Data Availability} \label{sec:materials}

Supporting software and executable notebooks for this work are available via Zenodo at TODO \citep{moreno2024hsurf}.
DStream algorithm implementations are also published on PyPI in the \texttt{downstream} Python package, where we plan to conduct longer-term, end-user-facing development and maintenance \citep{moreno2024downstream}.
All accompanying materials are provided open-source under the MIT License.

This project benefited significantly from open-source scientific software \citep{2020SciPy-NMeth,harris2020array,reback2020pandas,mckinney-proc-scipy-2010,waskom2021seaborn,hunter2007matplotlib,moreno2023teeplot}.
