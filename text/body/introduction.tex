\section{Introduction} \label{sec:introduction}

The field of evolution, whether biological or digital, often involves the study of a large group of organisms and their genetic material. A common question during these studies is how closely organisms are related to one another, and through phylogenetic analysis, ancestry trees can be built that outline the organisms' evolutionary history. These trees have countless applications throughout the field, emphasizing the importance of efficient and accurate methods to reconstruct them.

Phylogenetic analyses allow for characterizing and quantifying certain evolutionary processes, allowing researchers to make conclusions about the way a population evolved over time with varying degrees of accuracy depending on the method. For example, fitness parameters such as growth rate, probability of survival, and so on \citep{genthon2023cell}. Through large-scale analyses, patterns of evolutionary dynamics can be inferred, such as the effects of beneficial mutations on a population with varying levels of frequency \citep{levy2015quantitative}. On the other hand, one may want to study the rate at which particular ancestor species split into many new species -- the speciation rate -- as well as the rate at which species die out -- the extinction rate. By studying reconstructed phylogenies, both of these results can be determined \citep{stadler2013recovering}.

Phylogenetic analysis is also cruicial in the field of epidemiology, which becomes urgent in the face of pandemics such as COVID-19. Through such methods, we could determine which clade a particular strain came from, enabling the pinpointing of where and how a particular person was infected -- potentially leading to more efficient disease control \citep{wang2020role}. In another case, researchers could find relationships between different variants of the disease, showing that the Omicron variant was very distant from other variants \citep{kandeel2021omicron}.

\subsection{Phylogenies \& Digital Evolution}

Often, studying evolution through biological means is not as feasible, as laboratory experiements may take years, or even decades, to complete. Therefore, by simulating the behavior of a population, experiements can instead be done digitally, with simulations running in a fraction of the time. These systems can model key characteristics of biological populations, such as facilitation, movement, predation, and more. So, due to the nature of these simulations, conclusions about digital evolution can even be generalized to biological evolution \citep{dolson2021digital}.

Since they use similar mechanisms as biological evolution, organisms that have evolved digitally can also be analysed through phylogenies. One useful metric to determine if a population is likely to be successful is biodiversity, and digital populations are no exception. In fact, by using phylogenetic diversity (as opposed to other methods such as phenotypic diversity), stronger conclusions could be made about a digital population's fitness \citep{hernandez2022phylogenetic}.

Digital evolution can also be performed in a manner that allows the testing of phylogenetic methods that are applicable to biological evolution. The Aevol\_4b system, for instance, uses a genetic system corresponding to that of DNA, allowing any genetic information to be processed using methods directly from bioinformatics \citep{daudey2024aevol}.

\subsection{Reconstructing Phylogenies}

TODO