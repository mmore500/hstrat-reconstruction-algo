\section{Introduction} \label{sec:introduction}

The field of evolution, whether biological or digital, often involves the study of a large group of organisms and their genetic material. A common question during these studies is how closely organisms are related to one another, and through phylogenetic analysis, ancestry trees can be built that outline the organisms' evolutionary history. These trees have countless applications throughout the field, emphasizing the importance of efficient and accurate methods to reconstruct them.

A trivial application of phylogenetic analysis is to classify organisms based on their genetic related-ness. Often, morphological studies are not enough to determine if organisms are related, as many traits may have been developed independently. A rigorous phylogenetic analysis allows accurate classification of organisms' ancestry \citep{abaza2020what}. On the other hand, one may want to study the rate at which particular ancestor species split into many new species -- the speciation rate -- as well as the rate at which species die out -- the extinction rate. By studying reconstructed phylogenies, both of these results can be determined \citep{stadler2013recovering}.

Phylogenetic analysis is also cruicial in the field of epidemiology. For example, phylogenetic analysis on the infamous COVID-19 virus could determine which clade a particular strain came from, enabling researchers to pinpoint where and how a particular person was infected \citep{wang2020role}.
