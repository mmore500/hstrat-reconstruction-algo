\section{Introduction} \label{sec:introduction}

Key aspects of the study of evolution, whether biological or digital, revolve around understanding the flow of genetic material among large populations of organisms.
As such, phylogenetic analyses assessing ancestry trees representing organisms' evolutionary histories are a core tool in evolutionary biology.

In biology, phylogeny estimations are typically reconstructed through \textit{post hoc} analysis of genetic similarities among organisms.
In contrast, direct, exact tracking at runtime is typical in \textit{in silico} experiments.
However, in memory-constrained parallel and distributed computing contexts, \textit{post hoc} reconstruction approaches can become advantageous owing to runtime synchronization and storage costs of direct tracking.

Akin to biological studies, efficacy of phylogenetic analysis in such digital experiments hinges on fast, accurate methods to estimate ancestry trees from genome data.
In this work, we present a novel trie-building algorithm that greatly reduces compute time necessary to reconstruct phylogenies from special-purpose markers on digital genomes, while producing results equivalent to a naive approach.

\subsection{Applications of Phylogenetic Analysis}

Phylogenetic analyses provide key means to characterize and quantify a broad array of evolutionary processes.
Classically, these analyses have been applied to investigation of species-level macroevolutionary dynamics revolving around speciation and extinction rates; however, population- and organism-level dynamics can also be inferred, such as the spread of beneficial mutations within a population or fitness parameters like growth rate and probability of survival \citep{genthon2023cell, levy2015quantitative, stadler2013recovering}.
Phylogenetic analysis is also crucial in the field of epidemiology, playing a key role in informing public health interventions.
In this context, phylogenetic methods can be used to determine transmission history, pinpointing where and how chains of infection unfold \citep{wang2020role}.
In this vein, phylogenies are also key in assessing the prevalence of  ``super-spreader'' dynamics wherein disease spread is driven by a small set of high-risk individuals \citep{colijn2014phylogenetic}.

% \subsection{Reconstructing Biological Phylogenies} \label{sec:introduction:bioreconst}

% In biological studies, phylogenetic reconstruction methods typically work by assessing nucleotide changes between aligned DNA sequences from sample organisms.
% Approaches include distance-based methods, where a distance matrix between organisms is computed and processed with methods such as neighbor-joining \citep{saitou1987neighbor}; or character-based methods, such as maximum-parsimony \citep{sober1991reconstructing}, which seeks to minimize the number of evolutionary changes necessary to explain an evolutionary history --- and maximum-likelihood \citep{felsenstein1981evolutionary}, which infers tree topologies maximizing a likelihood function \citep{de2014phylogenetic}.

\subsection{Phylogenies and Digital Evolution} \label{sec:introduction:digital}

In some cases, aspects of biological evolution can be difficult or infeasible to observe on human timescales; laboratory experiments may take years, or even decades, to complete \citep{wiser2013long,Stroud2025}.
By simulating the behavior of a population, some experiments can instead be conducted digitally --- often completing in a fraction of the time.
Digital experiments can model key characteristics of biological populations, such as variation, natural selection, ecological interactions, spatial distribution, and more \citep{dolson2021digital,haller2023slim}.
As such, conclusions from digital evolution experiments can contribute meaningfully to understanding biology \citep{pennock2007models}.

In digital evolution contexts, phylogenies have likewise proven valuable.
In application-oriented contexts, phylogeny-based biodiversity metrics have been shown as predictive of solution quality outcomes for evolutionary computation-based optimization \citep{hernandez2022phylogenetic}.
Phylogeny-based methods can also be applied to characterize more general aspects of ecology, spatial structure, and selection pressure within \textit{in silico} populations \citep{moreno2023toward}.

Digital evolution approaches can also serve as a testbed to assess bioinformatics methodologies.
The Aevol\_4b system, for instance, uses a genetic system corresponding to that of DNA, allowing any genetic information to be processed using methods directly from bioinformatics \citep{daudey2024aevol}.
Likewise, population genetics work often incorporates SLiM, which supports sophisticated continuous-space modeling of single- and multi-species systems \citep{haller2023slim}.

Given the programmatic observability of digital simulations, digital evolution platforms typically incorporate direct tracking methods that record lineage ancestry as the simulation runs.
General-purpose phylogeny-tracking libraries exist for this purpose \citep{dolson2024phylotrack}, although many platforms simply incorporate bespoke implementations into their own software \citep{ofria2004avida}.

\subsection{Scaling Up Digital Evolution Experiments} \label{sec:introduction:distributed}

To achieve large-scale digital evolution experiments, it is necessary to move from a single-processor system to a more distributed approach with many computing units \citep{moreno2024trackable}.
In large-scale, many-processor simulations, however, challenges arise in managing a comprehensive record of ancestry.
To control memory use, it is typically necessary to trim away records of extinct lineages when performing direct tracking.
Detecting extinctions, however, introduces implementation complexity and overhead costs when lineage histories span across multiple processors.
Exhaustive tracking is also sensitive to data loss from crashed hardware or dropped messages, which has been highlighted as a key consideration in achieving very large-scale artificial life systems \citep{ackley2016indefinite,ackley2014indefinitely}.

Challenges associated with comprehensive tracking are especially acute in specialized hardware accelerator devices, which represent a promising emerging direction in high-performance computing \citep{emani2024democratizing}.
In incorporating thousands of processor cores per device, these hardware architectures impose trade-offs in memory capacity limitations and data locality restrictions that limit the feasibility of comprehensive tracking.
In such contexts, reconstruction-based approaches can provide an attractive balance between data fidelity and data collection overhead.

\subsection{Hereditary Stratigraphy} \label{sec:introduction:hstrat}

\begin{figure*}

\centering
\begin{minipage}{0.55\textwidth}

\begin{minipage}{0.4\linewidth}
\centering
\includegraphics[height=1.2in]{img/hstratschematic-evolve}
\subcaption{evolve}
\label{fig:hstratschematic:evolve}
\end{minipage}%
\centering
\begin{minipage}{0.2\linewidth}
~
\includegraphics[height=1.2in]{img/hstratschematic-sample}
\subcaption{sample}
\label{fig:hstratschematic:sample}
\end{minipage}%
\begin{minipage}{0.4\linewidth}
\centering
\includegraphics[height=1.2in]{img/hstratschematic-reconstruct}
\subcaption{reconstruct}
\label{fig:hstratschematic:reconstruct}
\end{minipage}
\end{minipage}%
~~
\begin{minipage}{0.43\textwidth}
\caption{%
\textbf{Overview of hereditary stratigraphy.}
\small
At runtime, genomes are annotated with randomly-generated heritable markers (panel \ref{fig:hstratschematic:evolve}).
To maintain fixed-memory footprint, some markers are overwritten.
Geonomes of interest are sampled at runtime and from end state (panel \ref{fig:hstratschematic:sample}).
Decoded genome markers enable estimation of evolutionary relatedness (panel \ref{fig:hstratschematic:reconstruct}), subject to error from marker-value collisions and discarded markers.
}
\label{fig:hstratschematic}
\end{minipage}

\end{figure*}

Under controlled conditions, such as laboratory experiments or evolution simulations, genetic material may be engineered to facilitate the accuracy and efficiency of estimating phylogenetic relatedness \citep{li2024reconstructing,ackley2023robust}.%
\footnote{Notably, Ackley has applied barcoding approaches to track recent ancestry among emergent replicators in a distributed fabric-computing context.}
Work developing hereditary stratigraphy (``hstrat'') methods seeks to operate analogously, providing techniques to organize genetic material in digital organisms that maximize reconstruction quality while minimizing memory footprint \citep{moreno2022hereditary}.
Hereditary stratigraphy components can be bundled with agent genomes in a manner akin to non-coding DNA (i.e., neutral with respect to agent traits and fitness), enabling generalizability across a wide variety of agent models.

Hereditary stratigraphy associates each generation along each lineage with an identifying ``fingerprint'' marker, referred to as a differentia.
On birth, each offspring receives a new differentia value and appends it to an inherited chronological record of past values --- corresponding to earlier generations along its lineage.
Under this scheme, mismatching differentia can be used to delimit the end of common ancestry between two organisms.
Figure \ref{fig:hstratschematic} summarizes this approach.

To save space, differentiae may be pruned away --- although, at the cost of reducing precision in inferring relatedness.
Using fewer bits per differentia can also provide many-fold memory savings; single bits or single bytes are appropriate for most use cases.

While inferring relatedness from biological sequence data can be a highly challenging and computationally-intensive problem \citep{miller2010creating},
the structured marker data used in hereditary stratigraphy somewhat ameliorates this challenge by allowing phylogeny reconstruction to be approached as a trie-building problem of identifying common string prefixes \citep{delabriandais1959file,moreno2024analysis}.
However, the presence of missing data due to some differentia being dropped to save memory complicates matters.

In the context of trie-building, missing marker time points (possessed by only a subset of organisms) effectively act as ``wildcard'' characters in prefix matching operations.
Therefore, placing an organism on a trie requires evaluating diverging string paths beyond the wildcard to identify further matches.
Given the likelihood of differentia value collisions for small differentia sizes (e.g., 1 bit), identifying the best-matching path after a wildcard value can require looking ahead several consecutive markers.
Furthermore, where consecutive wildcard values are encountered, the number of possible paths that must be explored can grow exponentially.

Although previous work has investigated the quality of phylogenies constructed from hereditary stratigraphy data using trie-based approaches \citep{moreno2025testing}, the computational intensity of the naive wildcard-matching approach has limited the scale of phylogenetic reconstructions investigated and restricted experimental throughput for smaller reconstructions.
Given the objective of hereditary stratigraphy methodology to facilitate studying very large-scale digital evolution experiments, achieving reconstruction efficiency sufficient for large-scale phyloanalysis is critical to the overall utility of the methodology in enabling observable experiments.

In this work, we describe an algorithm for efficient trie reconstruction in the face of missing data, and explore its performance characteristics.
The following section introduces our proposed ``shortcut'' algorithm for the trie building approach explored in this paper.
We then detail methods and results for benchmark trials assessing empirical scaling behavior and performance on large-scale billion-genome workloads.

% \citet{moreno2024trackable} used the naive trie building algorithms, and is the base for the benchmarking code that we did here (but we add some extensions here).
% The trees built in this work were only TODO tips.
%

% Earliest work with hereditary stratigraphy used a very simple UPGMA/neighbor joining approaches based on estimations of pairwise distance beteen taxa \citet{moreno2022hereditary}.

% The end result is the genetic information of an organism being a series of rank-differentia pairs, where each rank represents a generation at which a datum was retained, and the differentia representing the datum itself.
% However, the presence of missing information is a hurdle that a phylogenetic reconstruction algorithm must overcome, and may cause significant slowdowns using traditional algorithms.

% Given the various work that has been done on hereditary stratigraphy, it is useful to overview the recent work and to highlight how present work differs from it.

% \citet{moreno2022hereditary} introduced the general hereditary stratigraphy method, which was implemented in a accompanying software package \citep{moreno2022hstrat}.
% \citet{moreno2023toward} investigated the nature of phylostatistical fingerprints left by evolutionary forces of ecology, selection pressure, and spatial structure, and the effects of approximation error from hereditary stratigraphy on detecting these fingerprints.
% \citet{moreno2024methods} explored how the methodology might be extended from asexual systems, which we focus on here, to sexual systems through gene tree tracking and gene drive mechanisms.

% \citet{moreno2024structured} introduced the ``surface'' data structure which is used in this work, taking a generalized ring buffer approach to storing differentia data \citet{gunther2014algorithm}.
% The original trie-based reconstruction method is used in

% \textbf{TODO add more about hstrat}

% \subsection{Problem Statement} \label{sec:introduction:problem}

% TODO @mmore500

% Unfortunately, the existing naive algorithm is simply not fast enough for large-scale simulations, with


% @MAM I moved the following content into the "algorithm" section in the "naive trie building algorithm" subsection.
% I think what we want here is something that is going to be a tricky mix of vague and specific
% ... maybe a good term is "evocative"
% we want to evoke the core problem by describing "what" the challenge is rather than exactly "how" or "why"
% and we can end with "A more detailed description of the challenge solved in this work is provided in Section \ref{sec:algorithm}...
